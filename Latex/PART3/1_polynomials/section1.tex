\section{What is a Polynomial?}

A \textbf{polynomial} is a special type of math expression that involves numbers and variables. Here’s what you need to know:

\begin{itemize}
    \item \textbf{Variable:} A letter that stands for a number, like \( x \) or \( y \).
    \item \textbf{Coefficient:} A number that multiplies the variable, like 3 in \( 3x \).
    \item \textbf{Exponent:} A number that shows how many times the variable is multiplied by itself, like 2 in \( x^2 \).
\end{itemize}

A polynomial is made up of terms, which are parts of the expression separated by plus (+) or minus (\textminus) signs.

\section{Examples of Polynomials}

Here are some examples of polynomials:

\begin{itemize}
    \item \( 3x^2 + 2x - 5 \)
    \item \( 4y^3 - 7y + 1 \)
    \item \( -2x + 7 \)
\end{itemize}

In these examples:

\begin{itemize}
    \item \( 3x^2 + 2x - 5 \) has three terms: \( 3x^2 \), \( 2x \), and \( -5 \).
    \item \( 4y^3 - 7y + 1 \) has three terms: \( 4y^3 \), \( -7y \), and \( 1 \).
    \item \( -2x + 7 \) has two terms: \( -2x \) and \( 7 \).
\end{itemize}
