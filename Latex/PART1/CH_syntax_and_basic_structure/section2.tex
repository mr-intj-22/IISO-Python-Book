\section{Concepts}

\subsection{File Name}
A `Python` file (script) must have a name that follows these rules:
\begin{itemize}
    \item Ends with \textit{`.py'} which is the extension of Python files.
    \item Has all small letters (No capital letters).
    \item Should not have spaces (instead used underscore `\_`).
\end{itemize}

Examples of correct and incorrect scripts names.
\begin{table}[h!]
    \centering
    \begin{tabular}{|l|l|}
    \hline
    \textbf{Correct Python Script Name} & \textbf{Incorrect Python Script Name} \\ \hline
    my\_script.py & \textcolor{red}{my script.py} \\ \hline
    myscript.py & \textcolor{red}{my-script.py} \\ \hline
    myscript.py & \textcolor{red}{myscript} \\ \hline
    \end{tabular}
    \caption{Correct and Incorrect Python Script Names}
    \label{tab:script_name}
    \end{table}

\subsection{Indentation}
\begin{itemize}
    \item Importance of indentation in Python
    \item How to use spaces or tabs for indentation
    \item Examples of correct and incorrect indentation
\end{itemize}

\subsection{Comments}
\begin{itemize}
    \item Purpose of comments in code
    \item How to write single-line comments
    \item How to write multi-line comments
\end{itemize}

\subsection{Statements}
\begin{itemize}
    \item Definition of a statement
    \item Examples of different types of statements (e.g., print statements, assignment statements)
\end{itemize}

\subsection{Print Function}
\begin{itemize}
    \item Basic usage of the \texttt{print} function
    \item How to display messages and variables
    \item String formatting within the \texttt{print} function
\end{itemize}

\subsection{Code Blocks}
\begin{itemize}
    \item Definition and purpose of code blocks
    \item How to group statements into blocks
    \item Examples of code blocks in functions, loops, and conditionals
\end{itemize}

\subsection{Errors and Debugging}
\begin{itemize}
    \item Common syntax errors (e.g., indentation errors, missing colons)
    \item Basic techniques for debugging and fixing syntax errors
\end{itemize}

\subsection{File Execution}
\begin{itemize}
    \item How to run a Python script file from the command line
    \item Understanding script execution order
\end{itemize}

\subsection*{Key Terms}
\begin{itemize}
    \item \textbf{Syntax}: The set of rules that defines how to write code in Python (Grammar).
    \item \textbf{Indentation}: The use of 4 spaces or a tab at the beginning of a line to indicate a block of code.
    \item \textbf{Comment}: A line of text in the code that is not executed by the computer and is used to explain the code.
    \item \textbf{Print Function}: A function in Python used to display messages on the screen.
    \item \textbf{Statement}: A single line of code that performs an action.
    \item \textbf{Block of Code}: A group of statements that are grouped together, often by indentation.
\end{itemize}